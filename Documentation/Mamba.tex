\mainsection{The MAMBA Programming Language}
\label{sec:Mamba}
\subsection{Writing Programs}

Programs for the compiler are written in a restricted form of Python code,
with an additional library of functions and classes specifically for MPC
functionalities. The compiler then executes the Python code, and the MPC
library functions cause this to output
byte-code suitable for running on the virtual machine.
Whilst the compiler will, in theory, accept any valid Python
code as input, some language features may not work as expected.
The key point is that any native Python constructs are evaluated at compile
time, the effect being that all Python loops will be unrolled and all functions
inlined.
So to get the most out of the compiler and write efficient programs,
it is important to understand a little about the internals of the underlying implementation.

This section documents the library functions available to an MPC source code
file, and how to perform basic operations on the associated data types.
Basic knowledge of the workings of the Virtual Machine
(chapter \ref{sec:vm}) and the Python language are assumed. The definitions
of all library functions are given in \verb|library.py|.

\subsubsection{Data Types}
\label{ref:datatypes}

A complete set of all types and their methods can be found in
the html file in
\begin{center}
   \verb+$(HOME)/Documentation/Compiler_Documentation/index.html+
\end{center}
under the class \verb+types+.
Details of advanced algorithms can be found under the heading
\verb+Files+.
\subsubsection{Mod $p$ Data Types}
The compiler uses separate classes for clear and secret integer, fixed point
and floating point data types $\modp$. The integer classes \verb|cint| and \verb|sint|
correspond directly to a single clear or secret register in the virtual machine.
Integer types lie in the range $[-2^{k-1}, \dots, 2^{k-1}]$, for some
parameter $t$ with $p > 2^k$. Internally, these are represented in $\F_p$ by the
mapping $x \mapsto x \bmod{p}$, so will not wrap around modulo $2^k$ by default.
However, all clear values are reduced into the proper
range before comparison operations, to ensure correct results.
The parameters $k$ and $\kappa$ are chosen depending on the field size $x$ as follows:
\begin{itemize}
\item $x < 48 (bits)$ : $k = 24$
\item $48 \le x < 85 (bits)$ : $k = 32$
\item 128-bit field: $k = 64$
\end{itemize}
These choices allow some room for secure comparison with a statistical
security parameter of at least 30 bits, apart from the 32-bit field, which
should only be used for test purposes. More precisely, the default security parameter is chosen as follows:
\begin{itemize}
\item \verb|prime_bit_size < 48|:\\ 
Security parameter $\kappa$ is fixed to $6$ by Default.
\item \verb|48 < prime_bit_size  < 85|: \\
Security parameter $\kappa$ is fixed to $32$. 
\item \verb|prime_bit_size > 85|:\\
Default security parameter $\kappa$ is fixed to 40. Thus this will be the security parameter bit size for 128 bit finite
fields (which is the recommended prime size for MAMBA programs).
\end{itemize} 
The default values of $k$ and $\kappa$ can be modified using the commands described above if desired.

\subsubsection{Mod $2^n$ Data Types}
\label{sec:mod2n}
The compiler is also capable to process inputs that are encoded in a $\modn$  ring (like normal CPU registers) 
rather than $\modp$. 
Variables that instantiate these data types are then processed via Garbled Circuits. 

Data conversion between $\modp$ and $\modn$ types are also possible. As before, we make use of secret and clear data types. A clear integer can be represented by the \verb|regint| class, which works like normal 64-bit integer,
with the secure version being represented by the \verb|sregint| class.
Both types are used to represent secret shared/clear integers that are bounded by $2^{64}$. 
Both correspond directly to a single clear or secret register in the virtual machine. 
In these case both are integer types that lie in the range $[-2^{63}, \dots, 2^{63})$, both \verb|regint| and \verb|sregint| support inputs on such  range.

We have also included an extra data type called \verb|sbit|. 
This type can be instantiated via its constructor and also
by a result of a returned value from all \verb|sregint| comparison tests. 
It can be used to perform boolean operations on top of it, making it ideal for building complex logic predicates.

Moreover, the clear type \verb|regint| was designed in such a way that it can be used for other tasks. 
One example is an array index, since values used in range-loops and some of while-loops are \verb|regint|s. 
Also value in MemValue(described later) is stored as \verb|regint|. 
We use \verb|regint| in such a generic way, given that operations on regint are faster than on \verb|cint|. 
This is because \verb+regint+ values map directly to the integer values in the byte-codes,
whereas \verb+cint+ map to the clear type in the byte-code.

\paragraph{Conversions:} We can perform conversions between \verb|sint| and \verb|sregint|, bearing in mind that both have to be allocated in the same integer range. This can be done by simply instantiating either an \verb|sregint| or a
\verb|sint| object passing the corresponding register as a parameter in the constructor as follows: 
\begin{lstlisting}
a = sint(-5)
b = sregint(-5)
# casting
d = sregint(a)                        # returns signed sregint
c = sint(b)                           # returns signed sint
d = sregint(2**64-5)
e = sint.convert_unsigned_to_sint(saa) # Explicit unsigned conversion
\end{lstlisting}
The conversion back and forth between \verb|sregint| and \verb|sint| values uses the
daBit method of \cite{daBitPaper}, plus the evaluation (in the case of \verb|sint| to 
\verb|sregint| conversions) of a (relatively large) garbled circuit.
Thus whilst it is easy to pass from one secret data-type to the other, 
this operation has some computational and communication cost associated to it, hence 
we recommend such conversions to be used with care. 

~

\noindent{\bf sregint to sint conversion:}
Note that such a conversion may loose precision if 
$\log_2 p < 64$, thus if this is the case the user should guarantee
that the number being converted is in the correct range, otherwise
undefined behaviour will occur.
Recall a daBit is a doubly shared bit $b_i$ in two secret sharing
schemes (corresponding to our \verb|sint| and \verb|sbit| types)
$\sshare{b_i}_p$ and $\sshare{b_i}_2$.
To convert from a \verb|sregint| value to a \verb|sint| value we take
$64$ daBits $r_i$ and add them into the bits representing the 64-bit
\verb|sregint| $x$.
Thus we form $\sshare{v_i}_2 \asn \sshare{x_i}_2 \oplus \sshare{r_i}_2$.
The value $v_i$ is then opened to all parties.
All parties can then locally produce the sharing as an \verb|sint|
of the bits $x_i$ by forming
\[  \sshare{x_i}_p \asn v_i + \sshare{r_i}_p - 2 \cdot v_i \cdot \sshare{r_i}_p. \]
From these bits we can reconstruct $x$ as an \verb|sint| via
\[ \sshare{x}_p \asn - \sshare{x_{63}}_p \cdot 2^{63} + \sum_{i=0}^{62} \sshare{x_i}_p \cdot 2^i. \]
The standard/default conversion method proceeds as above, by treating
the input \verb|sregint| value as a {\em signed} 64-bit integer.
We have an additional conversion routine which treats the input as an
{\em unsigned} 64-bit integer and produces the result
\[ \sshare{x}_p \asn \sum_{i=0}^{63} \sshare{x_i}_p \cdot 2^i. \]


~

\noindent{\bf sint to sregint conversion:}
To convert from a \verb|sint| to a \verb|sregint| type is more complex.
We define two cases, one where $\log_2 p > 64$ and one where $\log_2 p < 64$.
Again in the latter case one needs to be careful that any values converted
fit in the correct range.

In the former case, We first take either $t=\log_2 p$ daBits $r_i$ 
(when $64+\mathsf{conv\_stat\_sec} \ge \log_2 p$),
or $t=64+\mathsf{conv\_stat\_sec}$ daBits $r_i$ 
(when $64+\mathsf{conv\_stat\_sec}<\log_2 p$),
and then forming the integer $r = \sum r_i \cdot 2^i$
in both the $\modp$ and the $\modn$ worlds. 
The value $\sshare{r}_p$ is subtracted from the input $\sshare{x}_p$ in 
the $\modp$ world, and the value $x-r$ opened. 
This is then enterred into a garbled circuit in the $\modn$ world so as to add back on $r$. 
This circuit is designed so that we convert {\it signed}  integers to {\it signed}
integers, where in the $\modp$ world signing is by taking integers with a centred 
rounding modulo $p$.
Of course if you convert a value bigger than $64$ bits in the $\modp$ world over to the $\modn$
world then you will loose information.

For this method to be secure we need that the value $r$ generated above is 
either large enough to drown out a $64$-bit value, or is itself sufficiently close 
to uniformly random to be secure.
Thus we require one of the following three conditions on $p$ to be met:
\begin{enumerate}
\item $\log_2 p > 64+ \mathsf{conv\_stat\_sec}$.
\item $(p - 2^{\log_2 p}) / 2^{\log_2 p} < 2^{-\mathsf{conv\_stat\_sec}}$, i.e. $p$ is just larger than a power of two.
\item $(2^{\log_2 p}-p) / p < 2^{-\mathsf{conv\_stat\_sec}}$, i.e. $p$ is slightly less than a power of two.
\end{enumerate}
If these conditions are not met, then the conversion will abort at runtime.
The value \verb|conv_stat_sec| is by default equal to 40, but this can be modified in
\verb|config.h|.

In the latter case, when $\log_2 p < 64$, we first generate $\log_2 p$ random daBits
$r_i$ until the value $r = \sum_{i=0}^{\log_2 p} r_i \cdot 2^i$ is less than $p$.
Thus we can treat $r$ as a uniformly random value modulo $p$.
We then perform the operations above, opening $x-r$, and then adding back on $r$
using a Garbled Circuit and adjusting to get the sign correct.

~

\noindent
Extractions of bits from an \verb|sregint| register can be done as follows:
\begin{lstlisting}
sa=sregint(5424)
for j in range(32,0,-1):
  sf=sbit()
  bitsint(sf,sa,j)
  cc=sf.reveal()
  print_int(cc)
print_ln("")
\end{lstlisting}
Setting specific bits in an \verb|sregint| via a \verb|sbit| are done as follows:
\begin{lstlisting}
ans=sregint(0)
for j in range(0,32):
  temp=sregint()
  sintbit(temp,ans,sbb,j)
  ans=temp;
  cc=ans.reveal()
  print_int(cc)
  print_ln("")
\end{lstlisting}


\paragraph{On regint and clear data types:} As it was noted before, \verb|regint| offers support for a number of housekeeping tasks. Because of this fact, conversions between all clear types, including \verb|regint| are supported. Furthermore, all operations that involve a clear register with \verb|regint| would return a \verb|regint| register. This can be achieved as follows: 
\begin{lstlisting}
# casting
a = cint(5)
b = regint(a)
a = cint(b)

# basic  operations:
d = a + b     # returns regint
c =  b + a    # returns regint
\end{lstlisting}
This would also implicitly allow the compiler to include operations between \verb|regint| and secret $\modp$ types. However, this is not recommended, given that the disparity on bit-length support in both $\modp$ and $\modn$ types might cause incongruity among the answers. We advise users to use the conversion routines instead to transition between both type families.

\subsubsection{Complex Secret Data Types}
Fixed and floating point types can be instantiated using the \verb|sfix| and \verb|sfloat| classes, and are described in later part of documentation about Advanced Algorithms.

\subsubsection{Creating data}
\func{sint(value)}
Loads the integer or clear integer \verb|value| into a secret register and returns the register.

\noindent
\textbf{Example:}
\begin{lstlisting}
i = sint(5)
\end{lstlisting}
\begin{footnotesize}
Note: value type can be: regint, int, long, sregint, sint and cint.
\end{footnotesize}

\func{cint(value)}
Loads the integer \verb|value| into a clear register and returns the register.

\noindent
\textbf{Example:}
\begin{lstlisting}
i = cint(5)
\end{lstlisting}
\begin{footnotesize}
Note: value type can be: regint, cint, int and long.
\end{footnotesize}

\func{regint(value)}
Loads the integer \verb|value| into a \verb|regint| register and returns 
the register.

\noindent
\textbf{Example:}
\begin{lstlisting}
i = regint(5)
\end{lstlisting}
\begin{footnotesize}
Note: value type can be: regint, int, long.
\end{footnotesize}

\func{sregint(value)}
Loads the integer or clear integer \verb|value| into a secret register and returns the register.

\noindent
\textbf{Example:}
\begin{lstlisting}
i = sregint(5)
\end{lstlisting}
\begin{footnotesize}
Note: value type can be: regint, int, long, sregint, sint.
\end{footnotesize}

%\func{sfix(value, exponent, bit_length)}
\func{sfix(value)}
It instantiates  an \verb|sfix| register based on \verb|value|. In case \verb|value| is a publicly available value, it loads it as a secret shared fix point number. In case is a secret shared integer, it places \verb|value| to the mantissa \verb|v| of the \verb|sfix| instance. 

\noindent
\textbf{Example:}
\begin{lstlisting}
i = sfix(5.5)
\end{lstlisting}
\begin{footnotesize}
Note: value type can be of any clear or secret numeric type.
\end{footnotesize}

\func{cfix(value)}
It instantiates  an \verb|sfix| register based on \verb|value|. 
Note that \verb|value| is a publicly available value, it loads it as a secret fix point. 

\noindent
\textbf{Example:}
\begin{lstlisting}
i = cfix(5.5)
\end{lstlisting}
\begin{footnotesize}
Note: value type can be of any clear numeric type.
\end{footnotesize}


\func{sfloat(value)}
It instantiates  an \verb|sfloat| register based on \verb|value| and returns an \verb|sfloat| object. 
Its instantiation logic mimics its fixed point counterpart. 
Current implementation supports basic logical and arithmetic operations with all data types.
We describe the formatting later in this section.

\noindent
\textbf{Example:}
\begin{lstlisting}
i = sfloat(5.5)
\end{lstlisting}
\begin{footnotesize}
Note: value type can be: int, float, sint, sfloat and sfix.
\end{footnotesize}

\func{cfloat(value)}
It is the clear register counterpart of \verb|sfloat|.
It instantiates a \verb|cfloat| register based on \verb|value| and returns a \verb|cfloat| object. 
It mimics \verb|sfloat| number representation, 
and its main function is to serve as an interface when operating 
between \verb|sfloat| instances and clear types and registers.

\noindent
\textbf{Example:}
\begin{lstlisting}
i = cfloat(5.5)
\end{lstlisting}
\begin{footnotesize}
Note: value type can be: int, float, sint, sfloat and sfix.
\end{footnotesize}

\func{load_int_to_secret_vector(vector)}
Loads a list of integers \verb|vector| into a vectorized \verb|sint| and operates on the vector as in a single instance (vectorized instructions).

\noindent
\textbf{Example:}
\begin{lstlisting}
A= [1, 2, 3]
SA= load_int_to_secret(A)
print_ln("Values from A: %s", SA) # the output is 123.
\end{lstlisting}

\func{load_secret_mem(address) \\
load_clear_mem(address)}
Returns the value stored in memory \verb|address| of the according type. The value had to be previously stored on SCALE. Memory in this context refers to a data-storage from SCALE and not physical memory. Users select a memory address when storing data, and the same address needs to be used to extract it. The calls can be implemented as follows: 

\noindent
\textbf{Example:}
\begin{lstlisting}
i =cint(5)
i.store_in_mem(0)
ci= load_secret_mem(i)
%print_ln("Values from A: %s", ci) # the output is 5 and type cint.
\end{lstlisting}
\begin{footnotesize}
Note: address type can be: regint, int, long and cint. It does the same as functions below and can store any data-type.
Note: operation supported for sregint, regint, sint, cint, sfix and sfloat.
\end{footnotesize}

\func{x.store_in_mem(address)}
Stores register \verb|x| into a given address
of the appropriate memory type. This basically implies it can be later retrieved by a \verb|load_mem| instruction. Memory addresses are decided by the user and are  stored by the compiler on SCALE.

\noindent
\textbf{Example:}
\begin{lstlisting}
i =sint(5)
i.store_in_mem(0)
si= load_secret_mem(i)
print_ln("Value stored on memory address 0: %s", si.reveal()) # the output is 5.
\end{lstlisting}
\begin{footnotesize}
Note: address type can be: regint, int, long and cint.
Note: operation supported for sregint, sint, cint, sfix and sfloat.
\end{footnotesize}

\func{x.load_mem(address)}
Loads the value stored in memory \verb|address| to the register. 
The address is selected during the invocation of a \verb|store_in_mem| call. 

\noindent
\textbf{Example:}
\begin{lstlisting}
i =sint(5)
i.store_in_mem(0)
si= sint.load_mem(0)
print_ln("Value stored on memory address 0: %s", si.reveal()) # the output is 5.
\end{lstlisting}
\begin{footnotesize}
Note: address type can be: regint, int, long and cint.
Note: operation supported for sregint, regint, sint, cint, sfix and sfloat.
\end{footnotesize}

\func{x.reveal()}
Opens the value in a secret register and returns a clear data-type, 
also referred as register for the now publicly available register.

\noindent
\textbf{Example:}
\begin{lstlisting}
si= sint(5)
print_ln("Value in the clear from si: %s", si.reveal()) # the output is 5.
\end{lstlisting}
\begin{footnotesize}
Note: x type can be: sregint, sbit, sint, sfix, and sfloat, resulting in a
regint, regint, cint, cfit and cfloat respectively.
\end{footnotesize}

\subsubsection{Stack Operations}
Each stack within each virtual processor can be pushed to and popped from.
One can also access the current value of the stack pointer, and then
manipulate this so as to be able to view and alter arbitrary elements
in the stack via the means of the peek and poke instructions.
The basic operations can be described for the \verb+regint+ data type
as follows:
\begin{lstlisting}
one=regint(1)
two=regint(2)
three=regint(3)
regint.push(one)
regint.push(two)
sp=regint.getsp()
c=regint.peek(sp-1)
regint.poke(sp,three)
d=regint.pop()
e=regint.pop()
print_ln(' %s',c)
print_ln(' %s',d)
print_ln(' %s',e)
\end{lstlisting}
This will print the values one, then three, then one. 
To use other stacks one uses the prefixes, \verb+sregint+, \verb+cint+,
\verb+sint+ and \verb+sbit+.

\subsubsection{Operations on Data Types}
Most of the usual arithmetic operators ($+, -, *, /, \%, <<, >>$) can be used with
clear and secret integer types, implemented with operator overloading.
This extends also to the case where one operand is a Python integer and one
is a clear or secret register.
Exponentiation ($**$) is implemented on $\modp$ types for immediate exponents and
immediate base 2.
There are some general limitations in regards to these operations however, when operating over secret shared inputs. We enumerate the following:
\begin{itemize}
\item The modulo operation can only be computed for immediate powers
  of two and its supported by $\modp$ types exclusively.
\item Division by \verb|sint| registers is not possible. 
\item One can divide an  \verb|sregint| variable by another \verb|sregint| variable,
      or a \verb|regint| by a \verb|sregint|, or vice versa.
\item $>>$ is not implemented for \verb|cint| registers as the right operand. 
All other shift operations for $\modp$ types are implemented.
The left and right shift by integers are available for \verb|sregint| variables.
\end{itemize}
Boolean operations $\&, \wedge, |, \sim$ are also supported on clear registers
and on \verb|sregint| registers.
This is true as well for the special data type \verb|sbit|, 
which is classified as a $\modn$ type. 
The compiler returns \verb|sbit| registers when performing comparisons on \verb|sregint| values. 
Note that comparisons are performed using subtraction and then comparison to zero,
thus overflow errors can result from the subtraction if done at the limits of the range
of representable numbers.
In the case of clear data types, comparisons work as expected for integers of bit length $n$. Whereas for \verb|sbit|, they operate as in boolean algebra. Note that the $\&$ operand is also supported between \verb|sregint| and \verb|sbit| types 
(where it behaves as a bit wise AND of the bits in the \verb|sregint| input by the single bit in the
\verb|sbit| input).

Access to the direct comparison with zero operations can be done via
\begin{lstlisting}
sa=sregint(5424)
sb1=sbit()
ltzsint(sb1,sa)
sb2=sbit()
eqzsint(sb2,sa)
\end{lstlisting}
or
\begin{lstlisting}
sa=sregint(5424)
sb1=sa.ltz()
sb2=sa.eqz()
\end{lstlisting}


Note that upon compilation all of the above operators return a 
new register -- it is not possible to directly modify the value 
of a specific register from MAMBA.
For example, in the following code, the \verb|*| operator creates a
fresh clear register, and assigns the result of the multiplication into this.
The name \verb|c| is then bound to the new register, and the previous
register is no longer accessible. The reason for this is to simplify the
register allocation procedure, details of which are in section
\ref{sec:regalloc}.

When operating between different types, the result will be 
secret if one of the operands was a secret register. 
Additionally, as in any other conventional programming language, 
the returned type will correspond to the type of the strongest precision. 

The goal of providing clear registers, is to provide the means 
to the user to interact and operate with secret values. 
Our examples make use of secret registers, but as mentioned, 
We now provide some examples for some basic operations. 
All of these operations are supported also in between secret and clear registers.
As a cautionary note, although supported, 
multiplication between fixed and secret float registers 
might cause some loss of precision, hence discouraged.

\paragraph{On Operations between regint and clear data types:} As it was mentioned above, operations between \verb|regint| and clear types such as \verb|cint| are not directly recommended. The reason, as stated, is because of the discrepancy on bit-wise sizes between \verb|regint| (which is limited to 64 bits) and $\modp$ clear types, where their size depends on the prime $p$. These operations look as follows:
\begin{lstlisting}
a = cint(2**65) #will overflow
b = regint(3)
# overflown:
c = a + b # returns a regint
d = a * b # returns a regint
e = a - b # returns a regint
\end{lstlisting}
Basically, operations between these types incur on an implicit casting of the \verb|cint| operator (to \verb|regint|). This will cause an overflow when the bit-length of \verb|cint| register or of the result of such operation is greater than 64 bits.

\paragraph{Multiplication:}
As before, multiplication is supported for secret and non-secret, 
integer and fractional data types. They can be invoked as follows: 
\begin{lstlisting}
c = sint(12)
c = c * c
f = sfix(12)
f = f * f
d = c * f
g = sfloat(12)
g = g * g
h = c * g
i = f * g 
j = sregint(12)
k = j * j
l = j * 12
\end{lstlisting}
In this small example, we can see how to multiply among different and the same data types. 
As a result $c=12^2$, $f=12.0^{2}$, $d=12.0^{4}$, $g = 12.0^{2}$, 
$h= 12.0^{4}$, and $i=12.0^{4} $. Note that \verb|d| is of type \verb|sfix|, 
whereas \verb|g|, \verb|h| and \verb|i| are of type \verb|sfloat|. 

In the above when you  multiply two \verb|sregint| values together you only
get the lower $64$-bits back in the resulting \verb|sregint|.
If you want to access the resulting top {\em and} bottom words then you
should use
\begin{lstlisting}
sa=sregint(2**62+212111)
sb=sregint(2**62-313131)

sd, sc = sa.mul_2_sint(sb)
\end{lstlisting}


\paragraph{Additions and Substractions:}
We follow the same principle as before:
\begin{lstlisting}
c = sint(12)
c = c + c
f = sfix(12)
f = f - f
d = c + f
g = sfloat (12)
g = g - g 
h = c + g
i = f + g

j = sregint(12)
k = j + j
l = j + 12
\end{lstlisting}
As before in this case the type for \verb|d| is \verb|sfix|, 
whereas the type of \verb|g|, \verb|h| and \verb|i| is \verb|sfloat|

\paragraph{Division and Modulus:}
We first revise how to perform division and modulus operations for $\modp$ types. In that sense, the compiler can handle also division and modulus operations for such types. 
However, these are not generic operations. 
Let us start by showing some basic constructions for division: 
\begin{lstlisting}
c= sint(12)
c= c / 3
f = sfix(3)
f = f/2	
g = sfloat(3)
g = g/2	
\end{lstlisting}
The results for \verb|c|, \verb|f| and \verb|g| are $4$, $1.5$ and $1.5$. 
Indeed, this is a quite natural way to call division on integers and decimal types. 
But it has to be noted that the division on integers is constructed as a multiplication 
between the numerator and the multiplicative inverse of the denominator. 
This is because of the modulo arithmetic the protocols are built upon. 

Moreover,  modulus operations are indeed somewhat different as we may see: 
\begin{lstlisting}
c = sint(2)
d = sint(3)
c = c % 2
d = d % 2
\end{lstlisting}
The operations will return, in the case of \verb|c| the value 0 and for \verb|d| 
the value 1. 
As with divisions, there are some observations: 
modulo operations can only be performed to powers of 2. 
Furthermore, note that modulo operations cannot be performed 
on non-integer types: \verb|sfix| and \verb|sfloat|.

In regards of $\modn$ types, we have included a division instruction in
the runtime for two variables of type \verb|sregint|. 
The compiler then overloads this to also allow division in the compiler
of a \verb|sregint| by an \verb|regint| and division of a \verb|regint|
by a \verb|sregint|.
The operations, mimic what could be expected from signed integer division. 
The \verb|sregint| division  can be used as follows:
\begin{lstlisting}
c = sregint(12)
d = sregint(4)
e = c / d
f = c / 4
g = 12 / d
\end{lstlisting}
In this case, the results would be \verb|sregint| instances of $ e = 3$, $f = 3$
and $g=3$.

\paragraph{Shift operations:}
We have included bit shifts operations for our basic integer $\modp$ data types. 
Such shifts do not work on fractional types nor on any $\modn$ type. 
\begin{lstlisting}
c = sint(2)
d = c << 1
e = e >> 1 
\end{lstlisting}
In this case, the output of \verb|d| is 4 as expected and from \verb|e| we obtain 1. 
Note that bit shifts only work on integer data types.
Similarly shifts can be performed for \verb|sregint| variables.

\paragraph{Exponentiation:}
We also provide a built-in exponentiation operator for exponentiation over integer and 
fractional $\modp$ data types ($**$), when the base is secret shared. 
This overload provides a simple implementation for successive multiplications. 
We provide more comprehensive protocols for exponentiation of fractional inputs 
in the following sections. 
\begin{lstlisting}[mathescape]
a = sint(2)
b = cint(3)
f = sfix(1.5)

c = a**b 	#returns $\displaystyle 2^3$
f = f**2	#returns $\displaystyle 1.5^2$
\end{lstlisting}
\textbf{NOTE}: For the reasons explained later with respect to fractional types, 
the exponent has to be of a \textit{native Python integer} type. 
We invite the reader to read the Advance Protocols section, to see alternative
methods to compute the exponent on fractional data types.

One can also raise a \verb|sfix| value to another \verb|sfix| value using the 
function \verb|mpc_math.pow_fx|. However, when doing this the routine uses 
\verb|sfloat| variables internally, thus you need to ensure the two types
are parametrized correctly.
In particular you must have \verb|k| parameter for \verb|sfix| being greater than
the \verb|vlen| for \verb|sfloat|.

\paragraph{Comparisons (Inequality Tests):}
We have in-built operators for comparisons as well. 
Indeed, comparison of secret values is supported, 
and returns a secret output containing 0 (false) or 1 (true).
They work on both integer (either $\modp$ or $\modn$ registers) and fractional data types. 
In this sense, they can be used as follows: 

\noindent
\textbf{Example:}
\begin{lstlisting}
# mod p
a = sint(1)
b = sint(2)
c = a < b
d = sfloat(3)
f = a < d
# mod2n
h = sregint(1)
i = sregint(2)
j = h < i # sbit     
k = regint(2)
j = h < k # sbit     
\end{lstlisting}

\noindent
NOTE: When executing a comparison using an \verb|sregint| register, the return data type is going to be an \verb|sbit|. We can then build secret shared complex statements using boolean operators.

\paragraph{Boolean Operators for sbit and sregint:}
We have included 4 basic boolean operators ($\&, \wedge, |, \sim$) for \verb|sbit| types. 
They behave in the same way native boolean type. They are the product of the 
comparison tests performed on \verb|sregint| registers or can be instantiated
via their contructor.
We aim to give the user a way to build complex logic predicates for, say if-like constructions. The operators can be invoked as follows: 
\begin{lstlisting}
# mod2n
a = sregint(1)
b = sregint(2)
c = a < b # sbit    1
d = a > b # sbit    0
temp_bit = sbit(1)
c = c & temp_bit
e = c | d # or
f = c & d # and     
g = c ^ d # xor
h = ~ c   # negation
# special case and
i = c & b # sregint with value 2
\end{lstlisting}
As it was previously mentioned, the $\&$ operation has been overloaded, 
to support also operations in between \verb|sbit| and \verb|sregint| types. 
For this specific case, its behaviour is similar to a multiplication and it returns 
an \verb|sregint| register.

The equivalent operations on \verb|sregint| registers are bitwise operations on the 
shared 64-bit values. One can also perform such operations between a
\verb|sregint| and a \verb|regint| register.
\begin{lstlisting}
# mod2n
a = sregint(1)
b = sregint(2)
c = regint(2)
d = a & b
e = a & c
f = a | b
g = a | c
h = a ^ b
i = a ^ c
j = ~ a 
\end{lstlisting}

\paragraph{On Clear Data types:} Before a comparison of clear integers (\verb|cint|) is done,
all operands are reduced into the range $[-2^{t-1}, \dots, 2^{t-1}]$.
Note that the virtual machine has no boolean type,
so any comparison operation returns a clear register value of 0
(false) or 1 (true).

\noindent
\textbf{Secret $\modp$ Data types.}  
The bit length of the comparison operators defaults to the parameter $t$, 
which is set-up by the compiler based on the input modulus, 
but if a different length is required it can be specified with the following functions:

\begin{lstlisting}
x.less_than(y, bit_length, sec_param)
x.greater_than(y, bit_length, sec_param)
x.less_equal(y, bit_length, sec_param)
x.greater_equal(y, bit_length, sec_param)
x.equal(y, bit_length, sec_param)
x.not_equal(y, bit_length, sec_param)
\end{lstlisting}

\noindent
The following simple example is applicable to all these methods: 
\begin{lstlisting}
a= sint(2)
b= sint(1)
c= a.less_than(b,128,40)
\end{lstlisting}
The output in this case is $1$ as expected. The 2 last parameters are not obligatory.


\subsubsection{Loading preprocessing data and sources of randomness}

For some programs, data from the preprocessing phase of SPDZ may be required (as source of randomness). Note that some kinds of randomness can  be generated during the program's offline phase. The function is implemented over the \verb|sint| class and cannot be accessed through any other data-type. Randomness can be accessed in the following ways:

\func{sint.get_random_triple()}
Returns three secret registers $a, b, c$ such that $a\cdot b = c$.
\begin{lstlisting}
a,b,c =sint.get_random_triple()
print_ln("these 2 results are equal %s, %s", (a*b).reveal() c.reveal())
\end{lstlisting}
The code above will show in this case $c$ and the result of the multiplication of $a$ and $b$, that should be equal.
\func{sint.get_random_bit()}
Returns a secret value $b$, with value in $\{0, 1\}$. The function can be used as follows:
\begin{lstlisting}
b =sint.get_random_bit()
print_ln("the result is either 0 or 1 %s", b.reveal())
\end{lstlisting}
The code will get a secret shared random bit.
\func{sint.get_random_square()}
Returns two secret values $a, b$ such that $a^2 = b$. Let us see the following example: 
\begin{lstlisting}
a,b =sint.get_random_square()
print_ln("these 2 results are equal %s, %s", (a*a).reveal() b.reveal())
\end{lstlisting}
The code will output the value of $a \cdot a$ versus $b$,  which are equal values. 

\func{sint.get_random_int(nbits)}
\textbf{Parameters:}\\
	\verb|nbits|: bitsize of the secret shared randomness to be produced. Must be a native Python integer variable (not any MAMBA data-type).
The function returns a random integer of size \verb|nbits|. This does not come directly from preprocessed data, but instead loads \verb|nbits| random bits and uses these to create the random secret integer. The function can be used as follows:
\begin{lstlisting}
a =sint.get_random_bit(5)
print_ln("the result is smaller that 2^a %s", a.reveal())
\end{lstlisting}
The output is a bounded integer by $2^5$. Note that \verb|nbits| can be a public input by the parties. 

\subsubsection{Printing}
We provide the following functions to printout outputs:
\func{print_str(string, *args)\\
print_ln(string, *args)}
Both of the functions do the same thing, only difference is that print_ln adds $newline$ after execution. Arguments are inserted into string in places of $\%$s respectively. 

\noindent
\textbf{Example:}
\begin{lstlisting}
x = 13
y = cint(5)
z = sint(x)
print_ln("x = %s, y = %s, z = %s", x, y, z.reveal())
\end{lstlisting}
Will print x = 13, y = 5, z = 13.

~\\

\noindent
There are other member functions which perform printing as well, these are
\func{cfix.print_fix()\\
regint.print_reg()\\
cfloat.print_float()\\
cint.print_reg()}


\subsubsection{How to print Vectorized data}
Suppose we have two vectorized data types such as sint, sfix, or sfloat. After we have done some operations then we want to print them. We will demonstrate here with sints:

\begin{lstlisting}
n = 10
x = sint(13, size=n)
y = sint(25, size=n)
z = x * y # this is now 325 on each of the 10 slots

z_array = sint.Array(n) # allocate memory to copy z
z.store_in_mem(z_array.address) # now z_array is full of z's data

for i in range(n):
	print_str("%s ", z_array[i].reveal())
\end{lstlisting}
This might seem useless - why do we want to do the same multiplication but 10 fold? Well we can put different data in each slot by first dumping same length arrays to $x$ and $y$ and then multiplication is going to be faster due to SIMD.

\subsection{Advanced Data Type and Programming Explanation}
		
\subsubsection{class sfix}
The \verb|sfix| class is based on Catrina and Saxena's work on processing fixed point precision arithmetic within MPC~\cite{CS10}. Basically, we use an integer mapping to encode a rational element represented up to certain precision. 
Given integer values $v$, $\beta =2$ and $f$, in SCALE, we can represent a rational value as follows:
	\[
	  x \approx v \cdot \beta ^{f}.
	\]
The results might be slightly different given the truncation of the information contained by the number. You can think of $f$ as the bitwise precision for the given fixed point representation.

%We refer to the bit-size of $b$ as $k$, and assume $v$ is a integer on the $\{-2^{k-1},2^{k-1}\}$ interval. 
%The results might be slightly different given the truncation of the information contained by the number. 
%You can think of $f$ as the bitwise precision for the given fixed point representation.
	\paragraph{Data Components:}
	The following are the most important data stored by the class:
	\func{v}
	\textbf{Accessed by:} \verb|Default|.\\
	\textbf{Type:} \verb|sint|.\\
	Stores a register of type \verb|sint| on the $\{-2^k-1, 2^k-1\}$ interval, encoding of the rational original value.
	\func{f}
	\textbf{Accessed by:} \verb|Default|.\\
	\textbf{Type:} \verb|int|.\\
	Stores the bitwise bit precision of the value to be stored by the instance.
	\func{k}
	\textbf{Accessed by:} \verb|Default|.\\
	\textbf{Type:} \verb|int|.\\
	Defines the mapping space. Basically, we can map numbers  from $-2^{k-1}$ to $2^{k-1}$, such that $k-f \geq 0$ so that no overflow occurs.
	
	\paragraph {Special Operations:}

\func{sfix.set_precision(f, k = None)}
			\textbf{Accessed by:} \verb|Default|.\\
    		\textbf{Parameters:}
     		\begin{description}
	     		\item[-] \verb|f|: New bitwise precision  value.
	    		\item[-] \verb|k|: New bitwise k, interval. \verb|default: NONE|.
    		\end{description}
    		\textbf{Returns:} No return value. \\
     		\textbf{Description:}
				Let you change the precision for the \verb|sfix| type. 
                                The requirement is that $f<k$ and $2 \cdot k + \kappa < \log_2 p$.
                                By default the values $(f,k)=(20,41)$ are chosen internally in the compiler.
                                If you change the precision for \verb|sfix| you should also change that for
                                \verb|cfix| otherwise you can get some strange arithmetic behaviours. 
                                It is used to fix the default precision on \verb|types.py|.  \\
			\textbf{Example:}
				To change the precision of \verb|sfix|:
\begin{lstlisting}
fixed_f=20
fixed_k=41
sfix.set_precision(fixed_f, fixed_k)
\end{lstlisting}
     			
\func{sfix.load_int(v)}
			\textbf{Accessed by:} \verb|Default|.		 \\
			\textbf{Parameters:}
			\begin{description}
				\item \verb|v|: Integer value to load into a \verb|sfix| instance.
			\end{description}
			\textbf{Returns:} No return value. \\
			\textbf{Description:}
				 It is used to do explicit initialization of an \verb|sfix| value from any integer instantiation value. \\
	     	\textbf{Example:}
    		    To initialize a \verb|sfix| value with an integer:
\begin{lstlisting}
a = sfix()
a.load_int(5)
b = a*3.0
print_ln("the answer is %s", b.reveal())  #the output is 15   	
\end{lstlisting}	 
\func{sfix.conv()}
			\textbf{Accessed by:} \verb|Default|.		 \\
			\textbf{Parameters:} N/A \\
			\textbf{Returns:} \verb|sint| value corresponding to the mantissa (\verb|v| value) mapping. \\
			\textbf{Description:}
				Function obtains the mantissa (\verb|v| value) mapping of the sorted value by the instance. \\
	     	\textbf{Example:}
    		    To obtain the value of the mantissa:
\begin{lstlisting}
a =sfix()
a.load_int(4.5)
v = a.conv()
print_ln("the answer is %s", v.reveal())  # the output is 4718592   	
\end{lstlisting}	 
%\func{sfix.store_in_mem}
		
		
\func{sfix.sizeof()}
			\textbf{Accessed by:} \verb|Default|.		 \\
			\textbf{Parameters:} N/A \\
			\textbf{Returns:}
				 Python native \verb|int| value corresponding to the size of memory slots occupied by the instance. \\
			\textbf{Description:}
				It returns the global_vector_size times 1. \\
	     	\textbf{Example:}
    		    To obtain reciprocal you can execute:
\begin{lstlisting}
a =sfix()
a.load_int(4.5)
r = a.sizeof()
print_ln("the answer is %s", r)  # the output is 1. 
             # By Default the global_vector_size is set to 1.    	
\end{lstlisting}	 
\func{sfix.compute_reciprocal()}		
			\textbf{Accessed by:} \verb|Default|.		 \\
			\textbf{Parameters:} N/A \\
			\textbf{Returns:} \verb|sfix| value corresponding to the reciprocal of the instance. \\
			\textbf{Description:}
				It calculates and returns the reciprocal of an instance in secret shared form, 
				in whatever precision is supported by \verb|sfix|.\\
	     	\textbf{Example:}
    		    To obtain reciprocal you can execute:
\begin{lstlisting}
a =sfix()
a.load_int(4.5)
r = a.compute_reciprocal()
print_ln("the answer is %s", r.reveal())  # the output is 0.222222   	
\end{lstlisting}	 
	\paragraph{Observations:}
	\begin{description}
	\item[-] The class should not be initialized from a \verb|sint| object. 
    Application level invocations should use the function \verb|load_int|.
    \item[-] The default precision and mantissa size, for \verb|sfix|, are fixed assuming at least a 128 bit modulus. 
    The values $(f,k)=(20,41)$ are fixed directly on \verb|types.py| and where fixed taken size restrictions into account.
    Note that by default the internal parameter $kappa$ is fixed to $40$ bits. 
	\item[-] You might get weird results due to precision loss  when operating with numbers close to the maximum allowable values.
	\end{description}
	
\subsubsection{class cfix}
We also provide users with an equivalent data type for clear inputs. 
It represents rational numbers in the same way, using Catrina and Saxena's~\cite{CS10}.
Only this time, $v$, is \textbf{not} secret shared.  
This basically means that an instance of the same number, on \verb|sfix| and \verb|cfix|, 
would  encode the number in the same fashion, as long as they are using 
the same precision parameters.
Note that since \verb|cfix| values are held in the clear there is no notion
of a statistical security parameter $\kappa$ for \verb|cfix| values.

	\paragraph{Data Components:}
	The following are the most important data stored by the class:
	\func{v}
	\textbf{Accessed by:} \verb|Default|.\\
	\textbf{Type:} \verb|int|.\\
	Stores a register on the $\{-2^k-1, 2^k-1\}$ interval, encoding of the rational original value.
	\func{f}
	\textbf{Accessed by:} \verb|Default|.\\
	\textbf{Type:} \verb|int|.\\
	Stores the bitwise bit precision of the value to be stored by the instance.
	\func{k}
	\textbf{Accessed by:} \verb|Default|.\\
	\textbf{Type:} \verb|int|.\\
	Defines the mapping space. We can map numbers  from $-2^{k-1}$ to $2^{k-1}$, such that $k-f \geq 0$ so that no overflow occurs.
	
	
	\paragraph {Special Operations:}

\func{cfix.set_precision(f, k = None)}
			\textbf{Accessed by:} \verb|Default|.\\
    		\textbf{Parameters:}
     		\begin{description}
	     		\item[-] \verb|f|: New bitwise precision value.
	    		\item[-] \verb|k|: New bitwise k, interval. \verb|default: NONE|.
    		\end{description}
    		\textbf{Returns:} No return value. \\
     		\textbf{Description:}
				Let you change the precision for \verb|cfix|. The precision parameters should be in line with those of \verb|cfix|, given that this class is used to interface operations between public available values and \verb|sfix| instances.
        
                                By default the values $(f,k)=(20,41)$ are chosen internally in the compiler.
                                If you change the precision for \verb|cfix| you should also change that for
                                \verb|sfix| otherwise you can get some strange arithmetic behaviours.
                                It is used to fix the default precision on \verb|types.py|. \\
			\textbf{Example:}
				To change the precision of a \verb|cfix|:
\begin{lstlisting}
fixed_f=20
fixed_k=41
cfix.set_precision(fixed_f, fixed_k)
\end{lstlisting}
     			
\func{cfix.load_int(v)}
			\textbf{Accessed by:} \verb|Default|.		 \\
			\textbf{Parameters:}
			\begin{description}
				\item \verb|v|: Public integer value to load into this \verb|cfix| instance.
			\end{description}
			\textbf{Returns:} No return value. \\
			\textbf{Description:}
				 It is used to do explicit initialization of an \verb|cfix| value from any integer instantiation value. \\
	     	\textbf{Example:}
    		    To initialize a \verb|cfix| value with an integer:
\begin{lstlisting}
a = cfix()
a.load_int(5)
b = a*3.0
print_ln("the answer is %s", b)  #the output is 15   	
\end{lstlisting}	 
\func{cfix.conv()}
			\textbf{Accessed by:} \verb|Default|.		 \\
			\textbf{Parameters:} N/A \\
			\textbf{Returns:} \verb|cint| value corresponding to the mantissa (\verb|v| value) mapping. \\
			\textbf{Description:}
				Function obtains the mantissa (\verb|v| value) mapping of the sorted value by the instance. \\
	     	\textbf{Example:}
    		    To obtain the value of the mantissa:
\begin{lstlisting}
a =cfix()
a.load_int(4.5)
v = a.conv()
print_ln("the answer is %s", v)  # the output is 4718592   	
\end{lstlisting}	 
%\func{sfix.store_in_mem}
		
		
\func{cfix.sizeof()}
			\textbf{Accessed by:} \verb|Default|.		 \\
			\textbf{Parameters:} N/A \\
			\textbf{Returns:}
				 Python native \verb|int| value corresponding to the size of memory slots occupied by the instance. \\
			\textbf{Description:}
				It returns the global_vector_size times 1. \\
	     	\textbf{Example:}
    		    To obtain reciprocal you can execute:
\begin{lstlisting}
a =cfix()
a.load_int(4.5)
r = a.sizeof()
print_ln("the answer is %s", r)  # the output is 4. 
			 # By Default the global_vector_size is set to 1.    	
\end{lstlisting}	 
\func{cfix.compute_reciprocal()}		
			\textbf{Accessed by:} \verb|Default|.		 \\
			\textbf{Parameters:} N/A \\
			\textbf{Returns:} \verb|cfix| value corresponding to the reciprocal of the instance. \\
			\textbf{Description:}
				It calculates and returns the reciprocal of the instance, on whatever precision is supported by \verb|cfix|.\\
	     	\textbf{Example:}
    		    To obtain reciprocal you can execute:
\begin{lstlisting}
a =sfix()
a.load_int(4.5)
r = a.compute_reciprocal()
print_ln("the answer is %s", r)  # the output is 0.222222   	
\end{lstlisting}	 
	\paragraph{Observations:}
	\begin{description}
	\item[-] A class instance cannot be initialized directly via a secret shared input. 
	In case you are loading an \verb|cint| value, we recommend you to use \verb|load_int|. 
	You should work with a \verb|cfix|, as you would do with a \verb|sfix| type.
	\item[-]\verb|cfix| by default, uses the same precision and mantissa size as \verb|sfix|. 
	That means it requires atleast a 128 bit modulus. 
    The values $(f,k)=(20,41)$ are fixed directly on \verb|types.py| and where fixed taken size restrictions into account.
	\end{description}

\subsubsection{class sfloat}

\noindent
A floating point number $x$ is represented as 
\[
			x\approx (1-2 \cdot s)\cdot (1-z)\cdot v\cdot 2^{p}.
\]
where $s$ is the sign bit,
$z$ is a bit to signal a zero or not,
$v$ is the \textit{significand} or \textit{mantissa},
and $p$ is the \textit{exponent}.
We also maintain a flag $\err$ which determines whether
some form of error state in the floating point variable
has occured (e.g. underflow, overflow, division by zero, taking
square roots of negative numbers etc).
Note, like all floating point representations such an encoding is 
an approximation.
The reader should note that our encoding does not directly
emulate IEEE floating point standards, but tries to mimic them
(thus precision etc of results will be different from IEEE
standard).
Our encoding is in line with the one described in detail 
by Aliasgari et. al~\cite{ABZS13} and used across various 
complex protocols for mathematical operations on MPC.
The values $s$, $z$, $v$, $p$ and $\err$ are kept in
secret shared format.

We also maintain a statistical security parameter $\kappa$ 
associated with an \verb+sfloat+ which needs
to satisfy
\[ 2 \cdot \verb+vlen+  + \kappa < \log_2 p. \]
It is possible to change the default sizes for \verb|sfloat| values using the commands.
\begin{lstlisting}
sfloat.plen=5 
sfloat.vlen=10
sfloat.kappa=20
\end{lstlisting}
The default values being $8$, $24$ and $40$ respectively.

The $\err$ flag needs to be treated with some care.
For efficiency reasons this can be {\em any} integer, but only
a value of zero represents a valid number.
The flag is initialized to $0$ and this value would only change in case an error is produce. 
We propagate the error by adding the $\err$ flags of the operating instances.
Thus any positive value indicates an error, and also could, if revealed, reveal how
many errors have occured in a calculation.
Thus before revealing an \verb|sfloat| instance, we first obtain the bit $ b = (err == 0)$, 
and multiply it by the data components of the instance. So an \verb|sfloat| value
will equal zero if {\em any} error has occured.
That way we guarantee that the output would not leak any information 
related to the inputs or any intermediary value during the calculations. 

The \verb|sfloat| type supports basic arithmetic and logic operations with 
any clear or secret register, as well as standard data types. 
Public values, as well as standard types are implicitly cast to \verb|cfloat| 
before operating on an \verb|sfloat| instance. The result will always be of type \verb|sfloat|. 
It is not recommended to utilize fixed and floating point operations unless
precision loss is taken into account.
 
\paragraph{Data Components:}
The following is a detailed description of the most important member state variables of \verb|sfloat|, their behaviour, and properties:		
		\func{v}
		\textbf{Accessed by:} \verb|Default|.\\
		\textbf{Type:} \verb|sint|.\\
		Stores the mantissa/significand of the encoding in secret form.
		\func{p}
		\textbf{Accessed by:} \verb|Default|.\\
		\textbf{Type:} \verb|sint|.\\
		Stores the exponent of the encoding in secret form.
		\func{s}
		\textbf{Accessed by:} \verb|Default|.\\
		\textbf{Type:} \verb|sint|.\\
		Stores a $\{0,1\}$ value storing the sign of the instance, where $\sshare{0}$ means is positive and $\sshare{1}$ otherwise.		
		\func{z}
		\textbf{Accessed by:} \verb|Default|.\\
		\textbf{Type:} \verb|sint|.\\
		Stores a $\{0,1\}$ value to flag when the \verb|sfloat| instance's value is zero. 
		When this variable takes a value of $\sshare{0}$, it means that, the \verb|sfloat| instance is a non-zero value and $\sshare{1}$ otherwise.		
		\func{err}
		\textbf{Accessed by:} \verb|Default|.\\
		\textbf{Type:} \verb|sint|.\\
		Stores a register of type \verb|sint|, 
		that serves to flag whether the instance is reporting a numeric error. 
		When this variable takes a value of $\sshare{0}$, 
		it means that the \verb|sfloat| instance is valid and non-zero otherwise. 
		Although not recommended, 
		in case of an error, if this value is later opened, 
		it would disclose the number of chained operations executed on t
		op its operation tree since the error was produced.
		
\paragraph{Special Operations:}

%\func{sfloat.load_mem(address, mem_type)}
		
\func{sfloat.set_error(error)}
			\textbf{Accessed by:} \verb|Default|.		 \\
			\textbf{Parameters:}
			\begin{description}
				\item[]\verb|error|:  sets the default precision error. 
			\end{description}
			\textbf{Returns:} N/A. \\
			\textbf{Description:} Not to be confused with the 
$\err$ flag. The error, in this case, pertains to the precision of the representation which is initialized in 0. It can be increased as follows:
\begin{lstlisting}							
cls.error += error - cls.error * error
\end{lstlisting}
			The formulation follows the literature on this topic. \\
			\textbf{Example:}      
			To alter the error you can do the following:			
\begin{lstlisting}
a =sfix(1.5)
a.set_error(0.1)
\end{lstlisting}
		
\func{sfloat.convert_float(v,vlen,plen)}
		      \textbf{Accessed by:} \verb|Default|. \\
		      \textbf{Parameters:}
		      \begin{description}
			     \item[]\verb|v|: data publicly available value (numeric Python data-type) to be transformed into float representation.
			     \item[]\verb|vlen|: bit-lenght of the mantissa that will encode \verb|v|.
			     \item[]\verb|plen|: bit-lenght of the exponential encoding of \verb|v|.
		      \end{description}
		      \textbf{Returns:} A tuple composed by the $5$ data components of a \verb|sfloat|. 
		      It can be used to instantiate a new \verb|sfloat| obtect.\\
		      \textbf{Description:} Transforms and secret shares an input 
		      in plain text to our floating point representation. 
		      Note that both \verb|vlen| and \verb|plen| parameters should 
		      be sufficiently large to handle \verb|v|.\\
		\textbf{Example:}
		An examplle, using  would look as follows:
\begin{lstlisting}
float_value= sfloat(9999, 14 ,1 ,0 , 0) 
print_ln("The float representation of 9999 is %s", float_value.reveal())		
\end{lstlisting}
%\func {store_in_mem}
\func{sfloat.sizeof()}
                        \textbf{Accessed by:} \verb|Default|. \\
                        \textbf{Parameters:} N/A \\
                        \textbf{Returns:}
                                 Python native \verb|int| value corresponding to the size of 
                                 memory slots occupied by the instance. \\
                        \textbf{Description:}
                                Note that each \verb|sfloat| instance is composed by 5 
                                different values (v, p, z, s, and err), 
                                it returns the \verb|global_vector_size| times 5. 
                                In case the instance is vectorized, 
                                it would return the vector size time 5.\\
                \textbf{Example:}
                    To obtain reciprocal you can execute:
\begin{lstlisting}
a = sfloat(4.5)
r = a.sizeof()
print_ln("the answer is %s", r)  # the output is 5. 
			 # By Default the global_vector_size is set to 1.
\end{lstlisting}
                        
\subsubsection{class cfloat}
The main purpose of this class is to offer interoperability between public registers 
and Python numeric types, and instances of our secret \verb|sfloat| register. 
We use the same representation as it secret shared counterpart, 
this way we are able to perform operations in between \verb|sfloat| and publicly available values, 
once they are cast into \verb|cfloat| values.
This basically means that an instance of the same number, on \verb|sfloat| and \verb|cfloat|, 
would  encode the number in the same fashion, 
as long as they are using the same precision parameters. 
However, \verb|cfloat| values do not include an $\err$ parameter, 
given that such flag is not necessary.

\paragraph{Data Components:}
		The following is a detailed description of the most important member state variables of \verb|cfloat|, and how it stores its value representation:
		
		\func{v}
		\textbf{Accessed by:} \verb|Default|.\\
		\textbf{Type:} \verb|cint|.\\
		Stores the significand in clear form.
		\func{p}
		\textbf{Accessed by:} \verb|Default|.\\
		\textbf{Type:} \verb|cint|.\\
		Stores the  exponent in clear form.
		\func{s}
		\textbf{Accessed by:} \verb|Default|.\\
		\textbf{Type:} \verb|cint|.\\
		Stores a $\{0,1\}$ value storing the sign of the instance, 
		where $0$ means is positive and $1$ otherwise. 
		Our aim is to mimic the behaviour of its secret shared counterpart.
		\func{z}
		\textbf{Accessed by:} \verb|Default|.\\
		\textbf{Type:} \verb|cint|.\\
		Stores a $\{0,1\}$ value to flag when the \verb|cfloat| 
		instance's value is zero. 
		When this variable takes a value of $0$, it means that the \verb|cfloat| 
		instance is a non-zero value and $1$ otherwise. 
		Our aim is to mimic the behaviour of its secret shared counterpart.

\paragraph{Special Operations:}
\func{cfloat.sizeof()}
                        \textbf{Accessed by:} \verb|Default|. \\
                        \textbf{Parameters:} N/A \\
                        \textbf{Returns:}
                                 Python native \verb|int| value corresponding to the size of memory slots occupied by the instance. \\
                        \textbf{Description:}
                                Note that each \verb|sfloat| instance is composed by 4 
                                different values (v, p, z and s), 
                                it returns the \verb|global_vector_size| times 4. 
                                In case the instance is vectorized, 
                                it would return the vector size time 4.\\
                \textbf{Example:}
                    To obtain reciprocal you can execute:
\begin{lstlisting}
a =cfloat(4.5)             
r = a.sizeof()
print_ln("the answer is %s", r)  # the output is 4. 
			 # By Default the global_vector_size is set to 1.
\end{lstlisting}

\subsubsection{Branching and Looping}
As discussed earlier, all native Python loops and branches are evaluated at
compile time. So to allow for control flow breaks depending on register values
that are not known at compile time, and also to prevent unrolling of loops
(reducing code size), the special
library functions for looping and branching should be used.
Because of the constraints of the compiling process, there are several
possibilities for both branching and looping.
Some of them require that the body of a loop or branch is written
in a separate function. The loop or branch is then initiated by a special
function call, taking the loop or branch function as a parameter.

\paragraph{Branching:}
There are two ways of doing the if statement: 
\begin{center}
\begin{tabular}{|l|l|} \hline
\textbf{if_statement(condition, if_true, if_false=None)} & \textbf{if_then(condition)} \\ \hline
def if_true():                                           & c = cint(1)                 \\
\hspace{5 mm}print_ln("True")                            &                             \\
                                                         & if_then(c)                  \\
def if_false():                                          & print_ln("True")            \\
\hspace{5 mm}print_ln("False")                           & else_then()                 \\
                                                         & print_ln("False")           \\
c = cint(1)                                              & end_if()                    \\
if_statement(c, if_true, if_false)                       &                             \\ \hline
\end{tabular}
\begin{footnotesize}
\\ Note that the functions in if_statement do not need to be named as above, but a clear naming convention like this is recommended to avoid ambiguity.
\end{footnotesize}
\end{center}
In that case value in register \verb|c| is non-zero (1), so in both cases \verb|True| will be printed in the terminal. More complex if/else statements (e.g. with multiple else conditions), can be created by simply nesting further \verb|if_statement| calls within \verb|if_false| in the left example, or adding another statement after \verb|else_then| in the right example. \\

\paragraph{Looping:}
All types of while-loops and range-loops with simple examples, like in other languages we can implement same function using different types of loops.
\func{do_loop(condition, loop_fn)}
Executes the code in \verb|loop_body| once, and continues to do so while the clear register returned by \verb|loop_body| is non-zero. Unlike with if and else statements, here the loop function must always return the loop condition register. Otherwise there is no way of knowing which register holds the condition after each iteration, since the original condition register cannot be modified: 
\begin{lstlisting}
def loop_body(cond):
    print_ln("%s", cond)
    return cond-1

t = cint(5)
do_loop(t, loop_body)
\end{lstlisting}
\begin{footnotesize}
Prints numbers from 5 to 1.
\end{footnotesize}

\func{while_loop(loop_body, condition, arg1, arg2, ...)}
Here \verb+condition+ has to be a function returning a clear integer, and \verb+arg1, arg2, ...+ are the initial arguments given to that
function. \verb+loop_body+ has to return the arguments given \verb+condition+ after every loop iteration. In addition, these arguments are given to \verb+loop_body+, which can be used to store the state between loop iterations. This is how the example looks like and also in a more conventional order:

\begin{center}
\begin{tabular}{|l|l|} \hline
\textbf{while_loop}                         & \textbf{@while_do}                \\ \hline
def loop_body(i):                           & @while_do(lambda x: x $<$ 5, 0)  \\
\hspace{5 mm}print_ln("\%s",i+1)            & def loop_body(i):                \\
\hspace{5 mm}return i+1                     & \hspace{5 mm}print_ln("\%s",i+1) \\
                                            & \hspace{5 mm}return i+1          \\
while_loop(loop_body, lambda x: x $<$ 5, 0) &                                  \\ \hline
\end{tabular}
\begin{footnotesize}
\\ Both prints numbers from 1 to 5.
\end{footnotesize}
\end{center}

\func{range_loop(loop_body, stop) \\ range_loop(loop_body, start, stop[, step])}
For-range loops can also be implemented in two ways, second way is not very different from clasic for loop, if start value is not declared it is automaticaly set to be 0, as on the example:

\begin{center}
\centering
\begin{tabular}{|l|l|} \hline
\textbf{range_loop}              & \textbf{@for_range}               \\ \hline
def loop_body(i):                & @for_range(5)                    \\
\hspace{5 mm}print_ln("\%s",i+1) & def loop_body(i):                \\
                                 & \hspace{5 mm}print_ln("\%s",i+1) \\
range_loop(loop_body, 5)         &                                  \\ \hline
\end{tabular}
\begin{footnotesize}
\\ Both prints numbers from 1 to 5.
\end{footnotesize}
\end{center}

\func{do_while(loop_body)}
Finally, it is the most traditional variant of a do-while loop. The loop is stopped when the return value is zero or false. However, variables declared outside the function cannot be modified inside the function, but they can be read. So in order to create finite loop, memory have to be used, MemValue is the easiest way to deal with memory:
\begin{center}
\begin{tabular}{|l|l|} \hline
\textbf{do_while}                        & \textbf{@do_while}                       \\ \hline
def loop_body():                         & m = MemValue(cint(0))                    \\
\hspace{5 mm}m.write(m+1)                &                                          \\
\hspace{5 mm}print_ln("\%s", m)          & @do_while                                \\
\hspace{5 mm}return m $<$ 5              & def loop_body():                         \\
                                         & \hspace{5 mm}m.write(m+1)                \\
m = MemValue(cint(0))                    & \hspace{5 mm}print_ln("\%s", m)          \\
do_while(loop_body)                      & \hspace{5 mm}return m $<$ 5              \\ \hline
\end{tabular}
\begin{footnotesize}
\\ Both prints numbers from 1 to 5.
\end{footnotesize}
\end{center}
Memory in a classical way can also be used, address in function is read only, same function once again:

\begin{center}
\begin{tabular}{|l|l|} \hline
\textbf{do_while}                        & \textbf{@do_while}                       \\ \hline
def loop_body():                         & address = program.malloc(1, 'c')         \\
\hspace{5 mm}m = cint.load_mem(address)  & store_in_mem(cint(1), address)           \\
\hspace{5 mm}print_ln("\%s", m)          &                                          \\
\hspace{5 mm}store_in_mem(m+1, address)  & @do_while                                \\
\hspace{5 mm}return m $<$ 5              & def loop_body():                         \\
                                         & \hspace{5 mm}m = cint.load_mem(address)  \\
address = program.malloc(1, 'c')         & \hspace{5 mm}print_ln("\%s", m)          \\
store_in_mem(cint(1),address)            & \hspace{5 mm}store_in_mem(m+1, address)  \\
do_while(loop_body)                      & \hspace{5 mm}return m $<$ 5              \\ \hline
\end{tabular}
\begin{footnotesize}
\\ Both prints numbers from 1 to 5.
\end{footnotesize}
\end{center}

\subsubsection{Arrays}
Arrays are made very similar to the other programming languages, using the array[index] indexing method. To declare array: 

\func{type.Array(size)}
Where size is just a compile-time known integer, type is the data type of single array element, for exaple sint or cint. Note that in previous versions, the array declaration was Array(size, type). This is still suported but consider it as deprecated.

\begin{lstlisting}
new_array = sint.Array(10)

@for_range(len(new_array))
def range_body(i):
    new_array[i] = sint(i+1)

@while_do(lambda x: x < 5, 0)
def while_body(i):
    print_ln("%s", new_array[i].reveal())
    return i+1
\end{lstlisting}
Declares new array of size 10, then fills it with values from 1 to 10, at the end prints first five values, so it prints numbers from 1 to 5. Note that the values of array can be modified inside the function, they are exacly like MemValue. \\
\func{type.Matrix(rows, columns)}
2D arrays are also implemented, but they are called matrices, matrix can be used in the same way as array:
\begin{lstlisting}
new_matrix = sint.Matrix(3,2)

@for_range(len(new_matrix))
def range_body(i):
    new_matrix[i][0] = sint(2*i)
    new_matrix[i][1] = sint(2*i+1)

@while_do(lambda x: x < 3, 0)
def while_body(i):
    @for_range(2)
    def range_body(j):
        print_ln("%s",new_matrix[i][j].reveal())
    return i+1
\end{lstlisting}
Matrix is just an array of arrays so length of matrix is just number of arrays inside, as it is shown on the example the first value of declaration is number of arrays inside, the second is length of an arrays(example prints numbers from 0 to 5).

\func{type.MultiArray([n_1,\dots,n_k])}
kD arrays are also implemented. They are meant to be just containers (retrieve and set data) - no fancy functions added on top of them such as assign_all, etc:

\begin{lstlisting}
n = 3
m = 4
p = 5
a = sint.MultiArray([n,m,p])
b = sint.MultiArray([n,m,p])
c = sint.MultiArray([n,m,p])

for i in range(n):
	for j in range(m):
		for k in range(p):
			a[i][j][k] = i + j + k
			b[i][j][k] = 2 * (i + j + k)
			c[i][j][k] = (a[i][j][k] + b[i][j][k])

# now c[i][j][k] = 3 * (i + j + k)
\end{lstlisting}


\paragraph{Short-circuit evalution:}
The following functions provide short-circuit evaluation, which means
that only necessary terms are executed, just as \verb+&&+ and
\verb+||+ in C or \verb+or+ and \verb+and+ in Python:

\func{and_(term1, term2, ...) \\
  or_(term1, term2, ...) \\
  not_(term)}
Since the direct specification of a term would trigger immediate
executiation and thus lose the short-circuit property, only functions
are permitted inputs:

\begin{lstlisting}
    def compare():
        return a < b
    and_(lambda: x < y, compare)
\end{lstlisting}
However, the functions should be combined directly because they
already output lambda functions:

\begin{lstlisting}
    not_(and_(lambda: x < y, lambda: a < b), lambda: c < d)
\end{lstlisting}
It is possible to use short-circuit evaluation for branching and
\verb+do_while+ loops but not \verb+do_loop+ loops because the
condition also represents the state in the latter case.

\begin{lstlisting}
if_then(and_(lambda: x < y, lambda: a < b))
...
end_if()

@do_while
def f():
    ...
    return and_(lambda: x < y, lambda: a < b)
\end{lstlisting}

\subsubsection{Multi-threading}

Creating and running multiple, concurrent tapes on the virtual machine is
supported via a simple threading interface. A thread is created by creating
an \verb|MPCThread| object referring to a function, which contains the code
to be executed by the thread.

\begin{lstlisting}
def func1():
    store_in_mem(sint(1),0)
    store_in_mem(sint(1),0)
    for i in range(8):
        a = load_secret_mem(i)
        b = load_secret_mem(i+1)
        store_in_mem(a+b, i+2)

t = MPCThread(func1, 't1')
t.start()
t.join()
\end{lstlisting}
\begin{footnotesize}
Stores secret Fibonacci numbers in first 10 places of memory. \\
\end{footnotesize}
Multiple threads can be run in parallel and sequentially, but threads cannot
themselves spawn threads as this is not yet supported by the virtual machine.

\subsubsection{Testing}

Testing of programs is done using the \verb|test.sh| script, and the library
functions \verb|test| and \verb|test_mem|. To test output, execution of the
library functions is emulated in Python by the \verb|test_result.py| script,
and compared with the actual output from running the virtual machine.
To trigger tests when running the script, calls to the following functions
must be inserted into the source code file.

\func{test(value, lower=None, upper=None)}
Tests the value of register \verb|value| by storing it to memory and then
reading the memory file output by the virtual machine. If \verb|lower| and
\verb|upper| are
not specified, checks that \verb|value| corresponds exactly to the emulated
result. Otherwise, checks that \verb|value| lies in the range
\verb|[lower,upper]|.

\func{test_mem(value, address, lower=None, upper=None)}
Tests whether the value in a given memory address is equal to the integer
\verb|value|. If \verb|lower| and \verb|upper| are specified, ignore \verb|value|
and instead check the result lies in the given range.

Emulation is not currently supported for all library functions. Loops and
branches must be tested manually using \verb|test_mem|, and any programs that
rely on reading or writing to or from main memory may not emulate correctly
either.

\subsubsection{SIMD Operations}
As explained in the section on byte-codes (Section \ref{sec:vm}) {\em most} byte-codes
can be vectorized.
In most cases this results in just a saving on resulting code size, and also
in cycles for processing the byte-codes. 
However for the \verb|STARTOPEN| and \verb|STOPOPEN| commands this
results also in a reduction in the number of rounds of communication needed.

Just as one can call the non-vectorized byte-codes directly from MAMBA, one
can also call the vectorized ones directly as well. See how a parallel
load and store from memory is done in the following example.
The \verb|ldms| byte-code is executed $n$ times in parallel by
changing it to \verb|vldms| and then giving the number of 
times this should be executed as the first argument.
\begin{lstlisting}
n = 100

A = Array(n, sint)
B = Array(n, sint)
C = Array(n, sint)

# assign some dummy values
for i in range(n):
    A[i] = sint(2 * i)
    B[i] = sint(3 * i)

a = sint(size=n)
b = sint(size=n)

vldms(n, a, A.address)
vldms(n, b, B.address)

c = a * b

vstms(n, c, C.address)

C[0].reveal().print_reg()
C[1].reveal().print_reg()
C[n / 2].reveal().print_reg()
C[n - 1].reveal().print_reg()
\end{lstlisting}
Notice the multiplication open is performed in parallel and executes
100 multiplications in one go. 
This single Mamba line is compiled down into byte codes as...
\begin{lstlisting}
vtriple 100, s100, s0, s200 # 0
vsubs 100, s400, s500, s0 # 2
vsubs 100, s300, s500, s100 # 4
vstartopen 100, 2, s300, s400 # 5
vstopopen 100, 2, c100, c200 # 6
vmulm 100, s300, s0, c100 # 7
vadds 100, s0, s200, s300 # 8
vmulm 100, s200, s100, c200 # 9
vadds 100, s100, s0, s200 # 10
vmulc 100, c0, c100, c200 # 11
vaddm 100, s0, s100, c0 # 12
\end{lstlisting}

Usage of the SIMD operations works in much the same as the above code snippet.
For example if you call a class function with no argument to determine the
vector length you need to also pass in the vector length you want, via
\verb|size=n|. Otherwise the SIMD vector length is deduced from the class
variable.
For example see the following code:
\begin{lstlisting}
n = 100

A = Array(n, sint)

# assign some dummy values
for i in range(n):
    A[i] = sint(2 * i)

a = sint(size=n)

vldms(n, a, A.address)

# Enter 100 private values from player 0
b=sint.get_private_input_from(0, size=n)

# Do some vector operations, including vector reveals
c = a * b
c.reveal_to(1)

d = a + b
d.reveal_to(2)

e=d.reveal()

f= a*e
f.reveal_to(0)

g=a+e
g.reveal_to(0)

h=a.reveal()

i=e*h
i.public_output()

j=e+h
j.public_output()

# Now play with some vector operations on the stacks
sint.push(a)
b=sint.pop(size=n)
b.reveal_to(25)

sint.push(a)

# We can even poke/peek vectors at a time with vectorized locations
sp=sint.getsp()
S=Array(n,regint)
for i in range(n):
    S[i] = sp-i
s=regint(size=n)
vldmint(n, s, S.address)

sint.poke(s,b)
c=sint.peek(s)
c.reveal_to(1)

\end{lstlisting}
Note, not all vectorized operations have been tested for correctness, if you
find a bug let us know and we will fix it.


%%% Local Variables: 
%%% mode: latex
%%% TeX-master: "main"
%%% End: 

