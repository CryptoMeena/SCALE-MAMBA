\mainsection{User Defined Local Functions}
\label{sec:Local}
As you write complex applications you will soon see the
need to execute local operations on each player which are
rather complex. For example this might be formatting data,
or performing some local arithmetic which does not need
to be done securely. 
Prior to v1.5 there was two ways of doing this:
\begin{enumerate}
\item Write the local function in MAMBA or SCALE byte-codes
directly. This is often both a pain from a programming point
of view, and also produced highly inefficient code.
\item Use the calling application to execute the local
operations, and then use the I/O class to interact
between the two. This requires people to have calling
applications, which any deployed application will have),
but which a quick experimental setup is unlikely to
bother with.
But it also requires expensive interaction between the
SCALE engine and the external application via the I/O
class.
\end{enumerate}
Having implemented the user defined circuits for Garbling
we decided to also, in a similar manner, add a third method
of implementing local functions; namely via user-defined
C++-code.
To see this in operation we suggest looking at the directory
\verb|src/Local/|, and looking at the files there. Here
we implement some basic linear algebra routines.
Look at those files whilst reading this explanation.

\subsection{Defining Local Functions in C++}
Every local function is one which involves no operations
between the parties.
It can thus be {\em any} function on clear data, and
only a {\em linear} function on private data.
A local function is defined using the C++ signature
\begin{lstlisting}
    void apply_Function(int instr,
                        vector<long> &Cr_out, vector<aBitVector> &Sr_out,
                        vector<gfp> &Cp_out, vector<Share> &Sp_out,
                        const vector<long> &Cr_inp, const vector<aBitVector> &Sr_inp,
                        const vector<gfp> &Cp_inp, const vector<Share> &Sp_inp);
\end{lstlisting}
We now explain the arguments in turn:
\begin{itemize}
\item
The variable \verb|instr| is the instruction number; which is akin to the
earlier circuit number for garbled circuits.
We reserve all numbers less than 65536 for use by the developers, leaving
you to define numbers greater than 65536.
Once again, if you have a good function which might be useful to others
please let us know.
\item
Next come the output vectors, one for each possible output type.
Note that each local function {\em should} check that only one of these 
vectors has non-zero size on entry to the routine. The vector with non-zero size must
correspond to the byte-code type which we explain below.
Note, this last bit is not checked, and if you get the wrong byte
code with the wrong output sizes you can expect the system to crash!
\item Finally come the four input vectors.
\end{itemize}
The local function is registered with an instruction number in the
system by adding the function pointer to the \verb|functions| map
in the file \verb|src/Local/Local_Functions.cpp|.
Must like user defined Garbled Circuits were added into the
system earlier.

\subsection{Defining Local Functions in the MAMBA/byte-code Language} 
On the byte-code side of system we have four instructions
\verb|LF_CINT|, \verb|LF_SINT|, \verb|LF_REGINT| and \verb|LF_SREGINT|.
The second part of this instruction contains the return type
of the method (and this must corresponds to the output vector
of the C++ function which has non-zero size.
The arguments to the instruction are as follows:
\begin{itemize}
\item Total number of arguments.
\item The local function number being called.
\item The size of the output vector.
\item The four sizes of the input vectors.
\item The list of output registers.
\item The list of input registers in the order as in the
above C++ function signature.
\end{itemize}

\subsection{BLAS Examples}
In our Basic Linear Algebra System (BLAS) we provide currently four
routines
\begin{center}
\begin{tabular}{c|l}
Instruction Number & Function \\
\hline
0 & cint n-by-k matrix A by cint k-by-m matrix B \\
1 & sint n-by-k matrix A by cint k-by-m matrix B \\
2 & cint n-by-k matrix A by sint k-by-m matrix B \\
3 & Row Reduction of a n-by-m cint matrix A \\
\end{tabular}
\end{center}
In these examples the dimensions are passed via the
\verb|Cr_inp| variables, with the actual matrices
then passed in the \verb|Cp_inp| and \verb|Sp_inp|
variables.
The matrices are packed into the vectors using a standard
row-wise configuration.
In the following code example (given in \verb|Programs/Local_test/|)
we illustrate this with the matrices
\[
  A = \left( \begin{array}{ccc}
  1 & 2 & 3 \\  
  4 & 5 & 6 \end{array} \right) \quad \quad
  B = \left( \begin{array}{cc}
  7 & 8 \\
  9 & 10 \\
  11 & 12 
  \end{array} \right).
\]
\begin{lstlisting}
n=2
l=3
m=2

# Mult the two matrices
#  A = [1,2,3;4,5,6]
#  B = [7,8;9,10;11,12]
# which should give us
#  C = [58,64; 139, 154]


Cr_inp=[regint(n), regint(l), regint(m)]
Cp_A=[cint(i+1) for i in range(n*l)]
Sp_A=[sint(i+1) for i in range(n*l)]
Cp_B=[cint(i+7) for i in range(l*m)]
Sp_B=[sint(i+7) for i in range(l*m)]

Cp_out=[cint() for _ in range(n*m)]

LF_CINT(0,n*m,3,0,n*l+l*m,0,*(Cp_out+Cr_inp+Cp_A+Cp_B))

print_ln("Final CC Product is...")
for i in range(2):
  for j in range(2):
    print_str('%s ', Cp_out[i*2+j])
  print_ln('')

Sp1_out=[sint() for i in range(n*m)]
Sp2_out=[sint() for i in range(n*m)]

LF_SINT(1,n*m,3,0,l*m,n*l,*(Sp1_out+Cr_inp+Cp_B+Sp_A))

print_ln("Final SC Product is...")
for i in range(2):
  for j in range(2):
    print_str('%s ', Sp1_out[i*2+j].reveal())
  print_ln('')

LF_SINT(2,n*m,3,0,n*l,l*m,*(Sp2_out+Cr_inp+Cp_A+Sp_B))

print_ln("Final CS Product is...")
for i in range(2):
  for j in range(2):
    print_str('%s ', Sp2_out[i*2+j].reveal())
  print_ln('')

GEr_inp=[regint(n), regint(l)]
GE_out=[cint() for _ in range(n*l)]
LF_CINT(3,n*l,2,0,n*l,0,*(GE_out+GEr_inp+Cp_A))

print_ln("Final Gauss Elim on A is...")
for i in range(2):
  for j in range(3):
    print_str('%s ', GE_out[i*l+j])
  print_ln('')
\end{lstlisting}
When compiling this code the four different local function calls
get translated into the following byte-codes
\begin{lstlisting}
LF_CINT 25,0,4,3,0,12,0,c9,c8,c7,c6,r1,r0,r2,c12,c16,c19,c11,c15,c14,c18,c21,c13,c17,c20,c10 
LF_SINT 25,1,4,3,0,6,6,s2,s5,s7,s1,r1,r0,r2,c18,c21,c13,c17,c20,c10,s0,s4,s14,s15,s3,s6 
LF_SINT 25,2,4,3,0,6,6,s4,s6,s0,s3,r1,r0,r2,c12,c16,c19,c11,c15,c14,s9,s11,s13,s8,s10,s12 
LF_CINT 20,3,6,2,0,6,0,c5,c4,c3,c2,c1,c0,r4,r3,c12,c16,c19,c11,c15,c14 
\end{lstlisting}







